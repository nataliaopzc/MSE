% !TeX root = ../Doc_especific_mse_anchovy.tex

A partir del diálogo en torno a los potenciales intereses que la Subsecretaría de Pesca y Acuicultura considera prioritarios, se definieron métricas que serán revisadas una vez que las estimaciones hayan evaluado la condición del recurso anchoveta norte. Lo anterior, se realizará en tres períodos de tiempo: corto (1-2 años), medio (3-7 años) y largo plazo (8-15 años). 

Para el criterio de la selección de un PM se pondrá atención en la revisión de largo plazo. Sin embargo, el corto y mediano plazo podrá ser revisado como parte del proceso de evaluación del comportamiento del PM. No obstante, los analistas expertos advierten que, dada las características biológicas de la historia de vida de la anchoveta norte, el corto plazo es imposible poder testear la robustez de una medida de desempeño ya que los plazos serían muy breves para lograr una estabilización del stock proyectado. Las medidas de desempeño fueron clasificadas en tres ejes prioritarios, estos son resumidos en la Tabla 4.

\begin{itemize}
    \item \textbf{Objetivos asociados a la sustentabilidad del recurso}
    \begin{itemize}
        \item Probabilidad de que el stock, tal como se representa en los modelos operativos individuales, se encuentre en la región verde (B $>$ 0.9 B$_{RMS}$ y F$<1.1$F$_{RMS}$) en el diagrama de fase o Kobe plot (Figura 4).
        \item Probabilidad de alcanzar una biomasa asociada al RMS (B $>$ 0.9 B$_{RMS}$).
        \item Probabilidad de no estar en un escenario de sobrepesca (F $<$ 1.1 F$_{RMS}$).
        \item Probabilidad de no encontrarse en la región de agotamiento (B $>$ 0.5B$_{RMS}$). 
    \end{itemize} 
    \item \textbf{Objetivos asociados a la estabilidad}
    \begin{itemize}
        \item Maximizar la CBA (captura media).
        \item Minimizar la variabilidad de la captura. La variabilidad de la captura no supere un cierto valor de referencia. 
    \end{itemize}
    \item \textbf{Objetivos asociados al PM actual}
    \begin{itemize}
        \item Evaluación de la tolerancia del error de implementación en el PM actual. Este objetivo busca evaluar cuántas veces la CBA del segundo hito es mayor que la CBA del primer hito.
    \end{itemize}
\end{itemize}

\begin{table}[h]
    \centering
    \caption{Medidas de desempeño definidas durante el taller por los miembros de la Subsecretaría de Pesca y Acuicultura.}
    \label{tab:tabla4}
    \begin{tabular}{|p{5.5cm}|p{6cm}|p{2cm}|}
        \hline
        \textbf{Objetivo de Manejo} & \textbf{Medida de desempeño} & \textbf{Categoría}  \\
        \hline
        Evitar que el stock se encuentre en sobreexplotación & P (B $>$ 0.9 B$_{RMS}$) & Biológica\\
        \hline
        Evitar que el stock se encuentre en sobrepesca & P (F $<$ 1.1 F$_{RMS}$) & Biológica \\
        \hline
        Evitar que el stock se encuentre en la zona de agotamiento  & P (B $>$ 0.5 B$_{RMS}$) & Biológica \\
        \hline
        El stock se encuentre en la región verde del diagrama de fase (Figura 4) & P (B $>$ 0.9 B$_{RMS}$ y F $<$ 1.1F$_{RMS}$) & Biológica \\
        \hline
        Maximizar la CBA & Captura promedio & Pesquera \\
        \hline
        Estabilidad en la CBA & PVariabilidad de la captura promedio entre años & Pesquera\\
        \hline
        Tolerancia de error de implementación en el PM actual & N° de veces en que CBA$_{H2} >$ CBA$_{H1}$ & Manejo \\
        \hline
    \end{tabular}
\end{table}
