% !TeX root = ../Doc_especific_mse_anchovy.tex

La SSPA es la institución responsable de la redacción de los términos técnicos de referencia (TTR) del proyecto de determinación del estatus y posibilidades de explotación biológicamente sustentables de anchoveta y sardina española, Región de Arica y Parinacota a la Región de Antofagasta, CBA al año 2024. Este proyecto señala un nuevo objetivo (V) que requiere explícitamente la participación de los expertos internacionales que desarrollan y mantienen el software openMSE y el uso de dicho software en la ejecución del mismo proyecto. 
\newline

openMSE es un paquete de software desarrollado en la plataforma R (R Core Team 2023), el cual es considerado como un paraguas, ya que contiene tres librerías principales para construir modelos operativos y simular la dinámica de una pesquería:

\begin{itemize}
    \item \textbf{MSEtool}: correspondiente al núcleo del paquete openMSE, el cual permite construir modelos operativos y simular la dinámica de una pesquería (Hordyk et al. 2023).
    \item \textbf{SANtool}: condicionar modelos operativos con datos y aplicar métodos de evaluación intensivos en datos (Huynh et al. 2023). 
    \item \textbf{DLMtool}: aplicar estrategias de manejo en situaciones limitadas en datos (Carruthers y Hordyk 2018).
\end{itemize}

Los paquetes están diseñados para hacer simulaciones de la dinámica de la pesquería y el estudio del desempeño de estrategias de manejo alternativas en  un ciclo cerrado lo más simples y eficientes posible. Estos paquetes han sido aplicados en una amplia gama de pesquerías, como por ejemplo la pesquería de merluza del Pacífico Sur (Hordyk y Newman, 2019).
\newline

Debido a que es un conjunto de paquetes de R, los usuarios de openMSE pueden beneficiarse de todas las ventajas proporcionadas por este entorno (por ejemplo, manejo efectivo de datos, una gran colección de herramientas para el análisis de datos, funciones gráficas para el análisis y visualización de datos). Los paquetes openMSE también tienen muy buena documentación, tienen gráficos de salida bien diseñados y están hechos para que sean accesibles para todos los usuarios (no es necesario tener un alto nivel de conocimiento de R). Estas características han hecho que el software sea atractivo para los administradores pesqueros chilenos. El hecho de que el software sea de código abierto y de fácil acceso a través de la red CRAN contribuye aún más a la facilidad de comunicación y transparencia necesarias para un proceso de EEM exitoso.
