% !TeX root = ../Doc_especific_mse_anchovy.tex

El Instituto de Fomento Pesquero (IFOP) es la institución de investigación encargada de brindar asesoramiento científico sobre los niveles de CBA que son consistentes con el objetivo del RMS. El modelo de evaluación que actualmente es usado con propósitos de manejo pesquero en Chile ocupa información biológica-pesquera del sur de Perú y norte de Chile. El modelo de evaluación base es estructurado a la edad con información en tallas, para lo cual se emplea una clave talla-edad dinámica en tiempo y por fuente de información, el modelo asume una escala temporal semestral con dos reclutamientos y dos desoves por año, debido al extenso período de desove (6 a 8 meses) y el rápido crecimiento observado a través de anillos diarios de otolitos (Cerna y Plaza, 2016). Además, incorpora las biomasas totales acústicas del sur de Perú y norte de Chile, la biomasa desovante estimada a través del método de producción diaria de huevos de Chile, los desembarques y estructuras de tamaños de las flotas comerciales para el sur de Perú y norte de Chile, y la abundancia a la talla del crucero acústico del norte de Chile. La evaluación de stock de anchoveta es actualizada dos veces al año, la primera ocurre en octubre de cada año (i) donde el stock es proyectado dos años con un supuesto en los reclutamientos futuros (4 semestres) y la segunda ocurre en marzo del año siguiente (i+1) con información actualizada completa del año anterior (i), el stock se proyecta un año incorporando una penalización en el último reclutamiento estimado por el modelo y el supuesto en los reclutamientos futuros (2 semestres).
\newline

Este procedimiento en el establecimiento de la CBA tiene un alto grado de incertidumbre. Esto se debe porque para la primera actualización, la CBA sólo dependen del supuesto de reclutamiento que es ingresado en la proyección del stock. Y para la segunda actualización, la CBA depende de la forma en que es penalizado el último reclutamiento estimado por el modelo de evaluación. Esta penalización ocurre por un fuerte patrón retrospectivo en los relutamientos que presenta el modelo de evaluación. Para la penalización del último reclutamiento se ocupan dos relaciones basadas en los reclutamientos históricos estimados por el modelo de evaluación y la biomasa de juveniles (<11.5 cm) que estima el crucero acústico del norte de Chile que se realiza a fines de cada año. Estas relaciones se diferencian según las condiciones ambientales (anomalías de la temperatura superficial del mar) que dominan la zona de estudio cuando se realiza el crucero acústico del norte de Chile.
\newline

El enfoque de EEM implica el desarrollo de un marco que considere el sistema de manejo pesquero en su totalidad, incluida la dinámica poblacional de los recursos y de las flotas, el esquema de recopilación de datos, el método de evaluación de poblaciones utilizado al proporcionar asesoramiento para el manejo pesquero, y cualquier regla de control de captura. El desarrollo del marco de EEM para la administración pesquera chilena es totalmente consistente con el enfoque precautorio de la FAO para la ordenación pesquera (Punt 2008). El desempeño de estrategias de manejo actual y candidatas para la anchoveta del norte de Chile, incluida aquella definida en el PM, no han sido aún formalmente evaluadas mediante una EEM. La regla de control que actualmente se utiliza en anchoveta del norte de Chile es modelo basada (Rademeyer et al 2007) y se encuentra definida en el PMP (Res. Ex. 1197/2018). Sin embargo, aún no se ha evaluado formalmente el desempeño de estrategias de manejo candidatas para esta pesquería.
\newline

El diseño preliminar para el estudio de la EEM de la anchoveta del norte de Chile consistió en i) identificar los principales ejes de incertidumbre biológica para el recurso anchoveta de manera de definir un conjunto acotado de modelos operativos referenciales (MO), ii) Identificar un conjunto de procedimientos de manejo (PM) pesquero y iii) identificar un conjunto adecuado de métricas de desempeño (MD). Estos tres principales puntos del diseño del EEM fueron acordados junto a los científicos del IFOP, administradores pesqueros de la SSPA y miembros de los CCT-PP. Este documento describe las especificaciones técnicas del proceso de implementación de EEM en la pesquería de anchoveta norte y se considera en permanente revisión, y fue desarrollado en el taller presencial “Evaluación de Estrategias de Manejo para la implementación del Enfoque Precautorio en Anchoveta norte en el Contexto de la LGPA” desarrollado en Valparaíso entre el 31 de julio al 4 de agosto del 2023.

