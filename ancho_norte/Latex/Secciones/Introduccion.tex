% !TeX root = ../Doc_especific_mse_anchovy.tex


En febrero del año 2023, se publicó en el Diario Oficial de Chile la Ley 20.657, Ley General de Pesca y Acuicultura (LGPA), que cambio el paradigma predominante de manejo pesquero en Chile, debido a que el estado de explotación de las principales pesquerías nacionales se encontraba en su mayoría sobreexplotadas e incluso algunas en estado de colapso, siendo urgente poner a la conservación y el uso sostenible de los recursos como uno de los ejes rectores de la regulación de la actividad pesquera. Entre estos cambios, fue la adopción del enfoque de precautorio para la pesca (FAO 1995), el enfoque ecosistémico para la pesca (FAO 2003) y el rendimiento máximo sostenible (RMS, Maunder 2008) basados en objetivos de manejo pesquero (Reyes et al, 2017). Con esta legislación, además se establecieron los comités científicos técnicos y de manejo pesquero (Reyes et al, 2017) y se confirió mayor peso al asesoramiento científico en el proceso de toma de decisiones para establecer los niveles de captura (Leal et al. 2010).
\newline

La LGPA introdujo el requisito de planes de manejo pesqueros obligatorios (PMP, Título II Párrafo 3$^\circ$ LGPA). Estos instrumentos de manejo vinculante deben especificar los objetivos, las metas y el período para reconstruir o mantener las poblaciones de peces al nivel de Rendimiento Máximo Sostenido (RMS) junto con las estrategias para alcanzar los objetivos y metas establecidos. El PMP para la anchoveta y sardina española de las Regiones Arica y Parinacota, Tarapacá hasta la Región de Antofagasta fue aprobado mediante Res. Ex. N$^\circ$ 1197 del 9 de abril del 2018. El PMP declara una serie de objetivos operacionales asociados a estándares de manejo (indicador y punto de referencia) con los cuales se debe medir el progreso de la pesquería como consecuencia de la aplicación de medidas y/o acciones de manejo. Estos objetivos fueron sistematizados en cuatro dimensiones, biológica, ecológica, económica y social.
\newline

Para el caso de la dimensión biológica, el PMP declara llevar y mantener el stock de anchoveta a un nivel que permita asegurar la sustentabilidad biológica del recurso. Para tal efecto, el objetivo N$^\circ$1.1 es llevar y mantener el recurso al RMS, y la medida de manejo asociada a este objetivo es establecer una captura biológicamente aceptable (CBA) basada en los puntos biológicos de referencia (PBR, Res. Ex. N$^\circ$ 291 del año 2015). Para el caso del stock de anchoveta del norte de Chile se establecieron proxies al RMS (Payá et al. 2014), los cuales fueron ratificados por el comité científico técnico de pequeños pelágicos (CCT-PP). Para la biomasa desovante al RMS (BDRMS) se estableció un valor igual al 50\% de la biomasa desovante virginal (BD0) y para la mortalidad por pesca al RMS (FRMS) se estableció aquella mortalidad por pesca que en el largo plazo produce el 55\% de la biomasa desovante por recluta (F55\%BDPR). La regla de control para este objetivo es aplicar una mortalidad por pesca constante al nivel del RMS.
\newline

La evaluación de la estrategia de manejo (EEM) es el uso de la simulación para evaluar el desempeño de una combinación de métodos de evaluación de poblaciones y reglas de control de captura (estrategias de ordenación) ante la incertidumbre de objetivos pre-acordados (Smith et al. 1999). El enfoque de EEM implica el desarrollo de un marco que considere el sistema de manejo pesquero en su totalidad, incluida la dinámica poblacional de los peces, el esquema de recopilación de datos, el método de evaluación de poblaciones utilizado al proporcionar asesoramiento para el manejo pesquero y cualquier regla de control de captura. El enfoque EEM es totalmente consistente con el enfoque precautorio de la FAO para la ordenación pesquera (Punt 2008). Sin embargo, aún no se ha evaluado formalmente el desempeño de las estrategias de manejo candidatas para la anchoveta del sur de Perú y el norte de Chile. En este escenario, los objetivos de una correcta EEM deben estar consensuados entre el equipo de científicos (IFOP), el sector pesquero (Comité de manejo) y el sector administrativo (SUBPESCA).

