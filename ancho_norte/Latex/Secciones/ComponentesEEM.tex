% !TeX root = ../Doc_especific_mse_anchovy.tex

\hspace{-20pt}\textbf{Modelos operativos (MO)}
\newline

Los modelos operativos contienen una descripción matemática del sistema pesquero, incluida la biología de la población de peces, el patrón histórico de explotación de las flotas pesqueras y los procesos de monitoreo empleados para recopilar los datos pesqueros. Los MO, también incluyen los supuestos para el proceso de monitoreo de los datos a futuro empleados en las proyecciones y cualquier error de implementación de las decisiones de manejo en las proyecciones. Un proceso de EEM incluye un conjunto de MO diferentes, cada uno de los cuales representa una hipótesis diferente sobre la posible dinámica de la pesquería. Los MO deben representar las principales fuentes de incertidumbres del sistema pesquero. De este modo la EEM permite identificar una estrategia de manejo que sea robusta a este rango de incertidumbre.
\newline

\hspace{-20pt}\textbf{Estrategia de manejo (EM)}
\newline

Las estrategias de manejo, también denominadas procedimientos de manejo, son un conjunto de reglas que convierten los datos pesqueros en recomendaciones de manejo (e.g. una captura biológicamente aceptable (CBA), una talla mínima de extracción o alguna combinación de reglas de control). El objetivo principal de la EEM es evaluar el desempeño de diferentes EM a fin de identificar un EM que sea robusto a la incertidumbre del sistema pesquero.
\newline

\hspace{-20pt}\textbf{Métricas de desempeño (MD)}
\newline

Las métricas o indicadores de desempeño se utilizan para evaluar el desempeño de los procedimientos de manejo. Las MD son indicadores cuantitativos que pueden ser calculados dentro del contexto de una EEM y de este modo se emplean para evaluar y comparar el desempeño de un conjunto de EM candidatas.
